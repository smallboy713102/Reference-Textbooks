\documentclass[11pt]{article}

    \usepackage[breakable]{tcolorbox}
    \usepackage{parskip} % Stop auto-indenting (to mimic markdown behaviour)
    
    \usepackage{iftex}
    \ifPDFTeX
    	\usepackage[T1]{fontenc}
    	\usepackage{mathpazo}
    \else
    	\usepackage{fontspec}
    \fi

    % Basic figure setup, for now with no caption control since it's done
    % automatically by Pandoc (which extracts ![](path) syntax from Markdown).
    \usepackage{graphicx}
    % Maintain compatibility with old templates. Remove in nbconvert 6.0
    \let\Oldincludegraphics\includegraphics
    % Ensure that by default, figures have no caption (until we provide a
    % proper Figure object with a Caption API and a way to capture that
    % in the conversion process - todo).
    \usepackage{caption}
    \DeclareCaptionFormat{nocaption}{}
    \captionsetup{format=nocaption,aboveskip=0pt,belowskip=0pt}

    \usepackage[Export]{adjustbox} % Used to constrain images to a maximum size
    \adjustboxset{max size={0.9\linewidth}{0.9\paperheight}}
    \usepackage{float}
    \floatplacement{figure}{H} % forces figures to be placed at the correct location
    \usepackage{xcolor} % Allow colors to be defined
    \usepackage{enumerate} % Needed for markdown enumerations to work
    \usepackage{geometry} % Used to adjust the document margins
    \usepackage{amsmath} % Equations
    \usepackage{amssymb} % Equations
    \usepackage{textcomp} % defines textquotesingle
    % Hack from http://tex.stackexchange.com/a/47451/13684:
    \AtBeginDocument{%
        \def\PYZsq{\textquotesingle}% Upright quotes in Pygmentized code
    }
    \usepackage{upquote} % Upright quotes for verbatim code
    \usepackage{eurosym} % defines \euro
    \usepackage[mathletters]{ucs} % Extended unicode (utf-8) support
    \usepackage{fancyvrb} % verbatim replacement that allows latex
    \usepackage{grffile} % extends the file name processing of package graphics 
                         % to support a larger range
    \makeatletter % fix for grffile with XeLaTeX
    \def\Gread@@xetex#1{%
      \IfFileExists{"\Gin@base".bb}%
      {\Gread@eps{\Gin@base.bb}}%
      {\Gread@@xetex@aux#1}%
    }
    \makeatother

    % The hyperref package gives us a pdf with properly built
    % internal navigation ('pdf bookmarks' for the table of contents,
    % internal cross-reference links, web links for URLs, etc.)
    \usepackage{hyperref}
    % The default LaTeX title has an obnoxious amount of whitespace. By default,
    % titling removes some of it. It also provides customization options.
    \usepackage{titling}
    \usepackage{longtable} % longtable support required by pandoc >1.10
    \usepackage{booktabs}  % table support for pandoc > 1.12.2
    \usepackage[inline]{enumitem} % IRkernel/repr support (it uses the enumerate* environment)
    \usepackage[normalem]{ulem} % ulem is needed to support strikethroughs (\sout)
                                % normalem makes italics be italics, not underlines
    \usepackage{mathrsfs}
    

    
    % Colors for the hyperref package
    \definecolor{urlcolor}{rgb}{0,.145,.698}
    \definecolor{linkcolor}{rgb}{.71,0.21,0.01}
    \definecolor{citecolor}{rgb}{.12,.54,.11}

    % ANSI colors
    \definecolor{ansi-black}{HTML}{3E424D}
    \definecolor{ansi-black-intense}{HTML}{282C36}
    \definecolor{ansi-red}{HTML}{E75C58}
    \definecolor{ansi-red-intense}{HTML}{B22B31}
    \definecolor{ansi-green}{HTML}{00A250}
    \definecolor{ansi-green-intense}{HTML}{007427}
    \definecolor{ansi-yellow}{HTML}{DDB62B}
    \definecolor{ansi-yellow-intense}{HTML}{B27D12}
    \definecolor{ansi-blue}{HTML}{208FFB}
    \definecolor{ansi-blue-intense}{HTML}{0065CA}
    \definecolor{ansi-magenta}{HTML}{D160C4}
    \definecolor{ansi-magenta-intense}{HTML}{A03196}
    \definecolor{ansi-cyan}{HTML}{60C6C8}
    \definecolor{ansi-cyan-intense}{HTML}{258F8F}
    \definecolor{ansi-white}{HTML}{C5C1B4}
    \definecolor{ansi-white-intense}{HTML}{A1A6B2}
    \definecolor{ansi-default-inverse-fg}{HTML}{FFFFFF}
    \definecolor{ansi-default-inverse-bg}{HTML}{000000}

    % commands and environments needed by pandoc snippets
    % extracted from the output of `pandoc -s`
    \providecommand{\tightlist}{%
      \setlength{\itemsep}{0pt}\setlength{\parskip}{0pt}}
    \DefineVerbatimEnvironment{Highlighting}{Verbatim}{commandchars=\\\{\}}
    % Add ',fontsize=\small' for more characters per line
    \newenvironment{Shaded}{}{}
    \newcommand{\KeywordTok}[1]{\textcolor[rgb]{0.00,0.44,0.13}{\textbf{{#1}}}}
    \newcommand{\DataTypeTok}[1]{\textcolor[rgb]{0.56,0.13,0.00}{{#1}}}
    \newcommand{\DecValTok}[1]{\textcolor[rgb]{0.25,0.63,0.44}{{#1}}}
    \newcommand{\BaseNTok}[1]{\textcolor[rgb]{0.25,0.63,0.44}{{#1}}}
    \newcommand{\FloatTok}[1]{\textcolor[rgb]{0.25,0.63,0.44}{{#1}}}
    \newcommand{\CharTok}[1]{\textcolor[rgb]{0.25,0.44,0.63}{{#1}}}
    \newcommand{\StringTok}[1]{\textcolor[rgb]{0.25,0.44,0.63}{{#1}}}
    \newcommand{\CommentTok}[1]{\textcolor[rgb]{0.38,0.63,0.69}{\textit{{#1}}}}
    \newcommand{\OtherTok}[1]{\textcolor[rgb]{0.00,0.44,0.13}{{#1}}}
    \newcommand{\AlertTok}[1]{\textcolor[rgb]{1.00,0.00,0.00}{\textbf{{#1}}}}
    \newcommand{\FunctionTok}[1]{\textcolor[rgb]{0.02,0.16,0.49}{{#1}}}
    \newcommand{\RegionMarkerTok}[1]{{#1}}
    \newcommand{\ErrorTok}[1]{\textcolor[rgb]{1.00,0.00,0.00}{\textbf{{#1}}}}
    \newcommand{\NormalTok}[1]{{#1}}
    
    % Additional commands for more recent versions of Pandoc
    \newcommand{\ConstantTok}[1]{\textcolor[rgb]{0.53,0.00,0.00}{{#1}}}
    \newcommand{\SpecialCharTok}[1]{\textcolor[rgb]{0.25,0.44,0.63}{{#1}}}
    \newcommand{\VerbatimStringTok}[1]{\textcolor[rgb]{0.25,0.44,0.63}{{#1}}}
    \newcommand{\SpecialStringTok}[1]{\textcolor[rgb]{0.73,0.40,0.53}{{#1}}}
    \newcommand{\ImportTok}[1]{{#1}}
    \newcommand{\DocumentationTok}[1]{\textcolor[rgb]{0.73,0.13,0.13}{\textit{{#1}}}}
    \newcommand{\AnnotationTok}[1]{\textcolor[rgb]{0.38,0.63,0.69}{\textbf{\textit{{#1}}}}}
    \newcommand{\CommentVarTok}[1]{\textcolor[rgb]{0.38,0.63,0.69}{\textbf{\textit{{#1}}}}}
    \newcommand{\VariableTok}[1]{\textcolor[rgb]{0.10,0.09,0.49}{{#1}}}
    \newcommand{\ControlFlowTok}[1]{\textcolor[rgb]{0.00,0.44,0.13}{\textbf{{#1}}}}
    \newcommand{\OperatorTok}[1]{\textcolor[rgb]{0.40,0.40,0.40}{{#1}}}
    \newcommand{\BuiltInTok}[1]{{#1}}
    \newcommand{\ExtensionTok}[1]{{#1}}
    \newcommand{\PreprocessorTok}[1]{\textcolor[rgb]{0.74,0.48,0.00}{{#1}}}
    \newcommand{\AttributeTok}[1]{\textcolor[rgb]{0.49,0.56,0.16}{{#1}}}
    \newcommand{\InformationTok}[1]{\textcolor[rgb]{0.38,0.63,0.69}{\textbf{\textit{{#1}}}}}
    \newcommand{\WarningTok}[1]{\textcolor[rgb]{0.38,0.63,0.69}{\textbf{\textit{{#1}}}}}
    
    
    % Define a nice break command that doesn't care if a line doesn't already
    % exist.
    \def\br{\hspace*{\fill} \\* }
    % Math Jax compatibility definitions
    \def\gt{>}
    \def\lt{<}
    \let\Oldtex\TeX
    \let\Oldlatex\LaTeX
    \renewcommand{\TeX}{\textrm{\Oldtex}}
    \renewcommand{\LaTeX}{\textrm{\Oldlatex}}
    % Document parameters
    % Document title
    \title{C2\_W2\_Assignment}
    
    
    
    
    
% Pygments definitions
\makeatletter
\def\PY@reset{\let\PY@it=\relax \let\PY@bf=\relax%
    \let\PY@ul=\relax \let\PY@tc=\relax%
    \let\PY@bc=\relax \let\PY@ff=\relax}
\def\PY@tok#1{\csname PY@tok@#1\endcsname}
\def\PY@toks#1+{\ifx\relax#1\empty\else%
    \PY@tok{#1}\expandafter\PY@toks\fi}
\def\PY@do#1{\PY@bc{\PY@tc{\PY@ul{%
    \PY@it{\PY@bf{\PY@ff{#1}}}}}}}
\def\PY#1#2{\PY@reset\PY@toks#1+\relax+\PY@do{#2}}

\expandafter\def\csname PY@tok@w\endcsname{\def\PY@tc##1{\textcolor[rgb]{0.73,0.73,0.73}{##1}}}
\expandafter\def\csname PY@tok@c\endcsname{\let\PY@it=\textit\def\PY@tc##1{\textcolor[rgb]{0.25,0.50,0.50}{##1}}}
\expandafter\def\csname PY@tok@cp\endcsname{\def\PY@tc##1{\textcolor[rgb]{0.74,0.48,0.00}{##1}}}
\expandafter\def\csname PY@tok@k\endcsname{\let\PY@bf=\textbf\def\PY@tc##1{\textcolor[rgb]{0.00,0.50,0.00}{##1}}}
\expandafter\def\csname PY@tok@kp\endcsname{\def\PY@tc##1{\textcolor[rgb]{0.00,0.50,0.00}{##1}}}
\expandafter\def\csname PY@tok@kt\endcsname{\def\PY@tc##1{\textcolor[rgb]{0.69,0.00,0.25}{##1}}}
\expandafter\def\csname PY@tok@o\endcsname{\def\PY@tc##1{\textcolor[rgb]{0.40,0.40,0.40}{##1}}}
\expandafter\def\csname PY@tok@ow\endcsname{\let\PY@bf=\textbf\def\PY@tc##1{\textcolor[rgb]{0.67,0.13,1.00}{##1}}}
\expandafter\def\csname PY@tok@nb\endcsname{\def\PY@tc##1{\textcolor[rgb]{0.00,0.50,0.00}{##1}}}
\expandafter\def\csname PY@tok@nf\endcsname{\def\PY@tc##1{\textcolor[rgb]{0.00,0.00,1.00}{##1}}}
\expandafter\def\csname PY@tok@nc\endcsname{\let\PY@bf=\textbf\def\PY@tc##1{\textcolor[rgb]{0.00,0.00,1.00}{##1}}}
\expandafter\def\csname PY@tok@nn\endcsname{\let\PY@bf=\textbf\def\PY@tc##1{\textcolor[rgb]{0.00,0.00,1.00}{##1}}}
\expandafter\def\csname PY@tok@ne\endcsname{\let\PY@bf=\textbf\def\PY@tc##1{\textcolor[rgb]{0.82,0.25,0.23}{##1}}}
\expandafter\def\csname PY@tok@nv\endcsname{\def\PY@tc##1{\textcolor[rgb]{0.10,0.09,0.49}{##1}}}
\expandafter\def\csname PY@tok@no\endcsname{\def\PY@tc##1{\textcolor[rgb]{0.53,0.00,0.00}{##1}}}
\expandafter\def\csname PY@tok@nl\endcsname{\def\PY@tc##1{\textcolor[rgb]{0.63,0.63,0.00}{##1}}}
\expandafter\def\csname PY@tok@ni\endcsname{\let\PY@bf=\textbf\def\PY@tc##1{\textcolor[rgb]{0.60,0.60,0.60}{##1}}}
\expandafter\def\csname PY@tok@na\endcsname{\def\PY@tc##1{\textcolor[rgb]{0.49,0.56,0.16}{##1}}}
\expandafter\def\csname PY@tok@nt\endcsname{\let\PY@bf=\textbf\def\PY@tc##1{\textcolor[rgb]{0.00,0.50,0.00}{##1}}}
\expandafter\def\csname PY@tok@nd\endcsname{\def\PY@tc##1{\textcolor[rgb]{0.67,0.13,1.00}{##1}}}
\expandafter\def\csname PY@tok@s\endcsname{\def\PY@tc##1{\textcolor[rgb]{0.73,0.13,0.13}{##1}}}
\expandafter\def\csname PY@tok@sd\endcsname{\let\PY@it=\textit\def\PY@tc##1{\textcolor[rgb]{0.73,0.13,0.13}{##1}}}
\expandafter\def\csname PY@tok@si\endcsname{\let\PY@bf=\textbf\def\PY@tc##1{\textcolor[rgb]{0.73,0.40,0.53}{##1}}}
\expandafter\def\csname PY@tok@se\endcsname{\let\PY@bf=\textbf\def\PY@tc##1{\textcolor[rgb]{0.73,0.40,0.13}{##1}}}
\expandafter\def\csname PY@tok@sr\endcsname{\def\PY@tc##1{\textcolor[rgb]{0.73,0.40,0.53}{##1}}}
\expandafter\def\csname PY@tok@ss\endcsname{\def\PY@tc##1{\textcolor[rgb]{0.10,0.09,0.49}{##1}}}
\expandafter\def\csname PY@tok@sx\endcsname{\def\PY@tc##1{\textcolor[rgb]{0.00,0.50,0.00}{##1}}}
\expandafter\def\csname PY@tok@m\endcsname{\def\PY@tc##1{\textcolor[rgb]{0.40,0.40,0.40}{##1}}}
\expandafter\def\csname PY@tok@gh\endcsname{\let\PY@bf=\textbf\def\PY@tc##1{\textcolor[rgb]{0.00,0.00,0.50}{##1}}}
\expandafter\def\csname PY@tok@gu\endcsname{\let\PY@bf=\textbf\def\PY@tc##1{\textcolor[rgb]{0.50,0.00,0.50}{##1}}}
\expandafter\def\csname PY@tok@gd\endcsname{\def\PY@tc##1{\textcolor[rgb]{0.63,0.00,0.00}{##1}}}
\expandafter\def\csname PY@tok@gi\endcsname{\def\PY@tc##1{\textcolor[rgb]{0.00,0.63,0.00}{##1}}}
\expandafter\def\csname PY@tok@gr\endcsname{\def\PY@tc##1{\textcolor[rgb]{1.00,0.00,0.00}{##1}}}
\expandafter\def\csname PY@tok@ge\endcsname{\let\PY@it=\textit}
\expandafter\def\csname PY@tok@gs\endcsname{\let\PY@bf=\textbf}
\expandafter\def\csname PY@tok@gp\endcsname{\let\PY@bf=\textbf\def\PY@tc##1{\textcolor[rgb]{0.00,0.00,0.50}{##1}}}
\expandafter\def\csname PY@tok@go\endcsname{\def\PY@tc##1{\textcolor[rgb]{0.53,0.53,0.53}{##1}}}
\expandafter\def\csname PY@tok@gt\endcsname{\def\PY@tc##1{\textcolor[rgb]{0.00,0.27,0.87}{##1}}}
\expandafter\def\csname PY@tok@err\endcsname{\def\PY@bc##1{\setlength{\fboxsep}{0pt}\fcolorbox[rgb]{1.00,0.00,0.00}{1,1,1}{\strut ##1}}}
\expandafter\def\csname PY@tok@kc\endcsname{\let\PY@bf=\textbf\def\PY@tc##1{\textcolor[rgb]{0.00,0.50,0.00}{##1}}}
\expandafter\def\csname PY@tok@kd\endcsname{\let\PY@bf=\textbf\def\PY@tc##1{\textcolor[rgb]{0.00,0.50,0.00}{##1}}}
\expandafter\def\csname PY@tok@kn\endcsname{\let\PY@bf=\textbf\def\PY@tc##1{\textcolor[rgb]{0.00,0.50,0.00}{##1}}}
\expandafter\def\csname PY@tok@kr\endcsname{\let\PY@bf=\textbf\def\PY@tc##1{\textcolor[rgb]{0.00,0.50,0.00}{##1}}}
\expandafter\def\csname PY@tok@bp\endcsname{\def\PY@tc##1{\textcolor[rgb]{0.00,0.50,0.00}{##1}}}
\expandafter\def\csname PY@tok@fm\endcsname{\def\PY@tc##1{\textcolor[rgb]{0.00,0.00,1.00}{##1}}}
\expandafter\def\csname PY@tok@vc\endcsname{\def\PY@tc##1{\textcolor[rgb]{0.10,0.09,0.49}{##1}}}
\expandafter\def\csname PY@tok@vg\endcsname{\def\PY@tc##1{\textcolor[rgb]{0.10,0.09,0.49}{##1}}}
\expandafter\def\csname PY@tok@vi\endcsname{\def\PY@tc##1{\textcolor[rgb]{0.10,0.09,0.49}{##1}}}
\expandafter\def\csname PY@tok@vm\endcsname{\def\PY@tc##1{\textcolor[rgb]{0.10,0.09,0.49}{##1}}}
\expandafter\def\csname PY@tok@sa\endcsname{\def\PY@tc##1{\textcolor[rgb]{0.73,0.13,0.13}{##1}}}
\expandafter\def\csname PY@tok@sb\endcsname{\def\PY@tc##1{\textcolor[rgb]{0.73,0.13,0.13}{##1}}}
\expandafter\def\csname PY@tok@sc\endcsname{\def\PY@tc##1{\textcolor[rgb]{0.73,0.13,0.13}{##1}}}
\expandafter\def\csname PY@tok@dl\endcsname{\def\PY@tc##1{\textcolor[rgb]{0.73,0.13,0.13}{##1}}}
\expandafter\def\csname PY@tok@s2\endcsname{\def\PY@tc##1{\textcolor[rgb]{0.73,0.13,0.13}{##1}}}
\expandafter\def\csname PY@tok@sh\endcsname{\def\PY@tc##1{\textcolor[rgb]{0.73,0.13,0.13}{##1}}}
\expandafter\def\csname PY@tok@s1\endcsname{\def\PY@tc##1{\textcolor[rgb]{0.73,0.13,0.13}{##1}}}
\expandafter\def\csname PY@tok@mb\endcsname{\def\PY@tc##1{\textcolor[rgb]{0.40,0.40,0.40}{##1}}}
\expandafter\def\csname PY@tok@mf\endcsname{\def\PY@tc##1{\textcolor[rgb]{0.40,0.40,0.40}{##1}}}
\expandafter\def\csname PY@tok@mh\endcsname{\def\PY@tc##1{\textcolor[rgb]{0.40,0.40,0.40}{##1}}}
\expandafter\def\csname PY@tok@mi\endcsname{\def\PY@tc##1{\textcolor[rgb]{0.40,0.40,0.40}{##1}}}
\expandafter\def\csname PY@tok@il\endcsname{\def\PY@tc##1{\textcolor[rgb]{0.40,0.40,0.40}{##1}}}
\expandafter\def\csname PY@tok@mo\endcsname{\def\PY@tc##1{\textcolor[rgb]{0.40,0.40,0.40}{##1}}}
\expandafter\def\csname PY@tok@ch\endcsname{\let\PY@it=\textit\def\PY@tc##1{\textcolor[rgb]{0.25,0.50,0.50}{##1}}}
\expandafter\def\csname PY@tok@cm\endcsname{\let\PY@it=\textit\def\PY@tc##1{\textcolor[rgb]{0.25,0.50,0.50}{##1}}}
\expandafter\def\csname PY@tok@cpf\endcsname{\let\PY@it=\textit\def\PY@tc##1{\textcolor[rgb]{0.25,0.50,0.50}{##1}}}
\expandafter\def\csname PY@tok@c1\endcsname{\let\PY@it=\textit\def\PY@tc##1{\textcolor[rgb]{0.25,0.50,0.50}{##1}}}
\expandafter\def\csname PY@tok@cs\endcsname{\let\PY@it=\textit\def\PY@tc##1{\textcolor[rgb]{0.25,0.50,0.50}{##1}}}

\def\PYZbs{\char`\\}
\def\PYZus{\char`\_}
\def\PYZob{\char`\{}
\def\PYZcb{\char`\}}
\def\PYZca{\char`\^}
\def\PYZam{\char`\&}
\def\PYZlt{\char`\<}
\def\PYZgt{\char`\>}
\def\PYZsh{\char`\#}
\def\PYZpc{\char`\%}
\def\PYZdl{\char`\$}
\def\PYZhy{\char`\-}
\def\PYZsq{\char`\'}
\def\PYZdq{\char`\"}
\def\PYZti{\char`\~}
% for compatibility with earlier versions
\def\PYZat{@}
\def\PYZlb{[}
\def\PYZrb{]}
\makeatother


    % For linebreaks inside Verbatim environment from package fancyvrb. 
    \makeatletter
        \newbox\Wrappedcontinuationbox 
        \newbox\Wrappedvisiblespacebox 
        \newcommand*\Wrappedvisiblespace {\textcolor{red}{\textvisiblespace}} 
        \newcommand*\Wrappedcontinuationsymbol {\textcolor{red}{\llap{\tiny$\m@th\hookrightarrow$}}} 
        \newcommand*\Wrappedcontinuationindent {3ex } 
        \newcommand*\Wrappedafterbreak {\kern\Wrappedcontinuationindent\copy\Wrappedcontinuationbox} 
        % Take advantage of the already applied Pygments mark-up to insert 
        % potential linebreaks for TeX processing. 
        %        {, <, #, %, $, ' and ": go to next line. 
        %        _, }, ^, &, >, - and ~: stay at end of broken line. 
        % Use of \textquotesingle for straight quote. 
        \newcommand*\Wrappedbreaksatspecials {% 
            \def\PYGZus{\discretionary{\char`\_}{\Wrappedafterbreak}{\char`\_}}% 
            \def\PYGZob{\discretionary{}{\Wrappedafterbreak\char`\{}{\char`\{}}% 
            \def\PYGZcb{\discretionary{\char`\}}{\Wrappedafterbreak}{\char`\}}}% 
            \def\PYGZca{\discretionary{\char`\^}{\Wrappedafterbreak}{\char`\^}}% 
            \def\PYGZam{\discretionary{\char`\&}{\Wrappedafterbreak}{\char`\&}}% 
            \def\PYGZlt{\discretionary{}{\Wrappedafterbreak\char`\<}{\char`\<}}% 
            \def\PYGZgt{\discretionary{\char`\>}{\Wrappedafterbreak}{\char`\>}}% 
            \def\PYGZsh{\discretionary{}{\Wrappedafterbreak\char`\#}{\char`\#}}% 
            \def\PYGZpc{\discretionary{}{\Wrappedafterbreak\char`\%}{\char`\%}}% 
            \def\PYGZdl{\discretionary{}{\Wrappedafterbreak\char`\$}{\char`\$}}% 
            \def\PYGZhy{\discretionary{\char`\-}{\Wrappedafterbreak}{\char`\-}}% 
            \def\PYGZsq{\discretionary{}{\Wrappedafterbreak\textquotesingle}{\textquotesingle}}% 
            \def\PYGZdq{\discretionary{}{\Wrappedafterbreak\char`\"}{\char`\"}}% 
            \def\PYGZti{\discretionary{\char`\~}{\Wrappedafterbreak}{\char`\~}}% 
        } 
        % Some characters . , ; ? ! / are not pygmentized. 
        % This macro makes them "active" and they will insert potential linebreaks 
        \newcommand*\Wrappedbreaksatpunct {% 
            \lccode`\~`\.\lowercase{\def~}{\discretionary{\hbox{\char`\.}}{\Wrappedafterbreak}{\hbox{\char`\.}}}% 
            \lccode`\~`\,\lowercase{\def~}{\discretionary{\hbox{\char`\,}}{\Wrappedafterbreak}{\hbox{\char`\,}}}% 
            \lccode`\~`\;\lowercase{\def~}{\discretionary{\hbox{\char`\;}}{\Wrappedafterbreak}{\hbox{\char`\;}}}% 
            \lccode`\~`\:\lowercase{\def~}{\discretionary{\hbox{\char`\:}}{\Wrappedafterbreak}{\hbox{\char`\:}}}% 
            \lccode`\~`\?\lowercase{\def~}{\discretionary{\hbox{\char`\?}}{\Wrappedafterbreak}{\hbox{\char`\?}}}% 
            \lccode`\~`\!\lowercase{\def~}{\discretionary{\hbox{\char`\!}}{\Wrappedafterbreak}{\hbox{\char`\!}}}% 
            \lccode`\~`\/\lowercase{\def~}{\discretionary{\hbox{\char`\/}}{\Wrappedafterbreak}{\hbox{\char`\/}}}% 
            \catcode`\.\active
            \catcode`\,\active 
            \catcode`\;\active
            \catcode`\:\active
            \catcode`\?\active
            \catcode`\!\active
            \catcode`\/\active 
            \lccode`\~`\~ 	
        }
    \makeatother

    \let\OriginalVerbatim=\Verbatim
    \makeatletter
    \renewcommand{\Verbatim}[1][1]{%
        %\parskip\z@skip
        \sbox\Wrappedcontinuationbox {\Wrappedcontinuationsymbol}%
        \sbox\Wrappedvisiblespacebox {\FV@SetupFont\Wrappedvisiblespace}%
        \def\FancyVerbFormatLine ##1{\hsize\linewidth
            \vtop{\raggedright\hyphenpenalty\z@\exhyphenpenalty\z@
                \doublehyphendemerits\z@\finalhyphendemerits\z@
                \strut ##1\strut}%
        }%
        % If the linebreak is at a space, the latter will be displayed as visible
        % space at end of first line, and a continuation symbol starts next line.
        % Stretch/shrink are however usually zero for typewriter font.
        \def\FV@Space {%
            \nobreak\hskip\z@ plus\fontdimen3\font minus\fontdimen4\font
            \discretionary{\copy\Wrappedvisiblespacebox}{\Wrappedafterbreak}
            {\kern\fontdimen2\font}%
        }%
        
        % Allow breaks at special characters using \PYG... macros.
        \Wrappedbreaksatspecials
        % Breaks at punctuation characters . , ; ? ! and / need catcode=\active 	
        \OriginalVerbatim[#1,codes*=\Wrappedbreaksatpunct]%
    }
    \makeatother

    % Exact colors from NB
    \definecolor{incolor}{HTML}{303F9F}
    \definecolor{outcolor}{HTML}{D84315}
    \definecolor{cellborder}{HTML}{CFCFCF}
    \definecolor{cellbackground}{HTML}{F7F7F7}
    
    % prompt
    \makeatletter
    \newcommand{\boxspacing}{\kern\kvtcb@left@rule\kern\kvtcb@boxsep}
    \makeatother
    \newcommand{\prompt}[4]{
        \ttfamily\llap{{\color{#2}[#3]:\hspace{3pt}#4}}\vspace{-\baselineskip}
    }
    

    
    % Prevent overflowing lines due to hard-to-break entities
    \sloppy 
    % Setup hyperref package
    \hypersetup{
      breaklinks=true,  % so long urls are correctly broken across lines
      colorlinks=true,
      urlcolor=urlcolor,
      linkcolor=linkcolor,
      citecolor=citecolor,
      }
    % Slightly bigger margins than the latex defaults
    
    \geometry{verbose,tmargin=1in,bmargin=1in,lmargin=1in,rmargin=1in}
    
    

\begin{document}
    
    \maketitle
    
    

    
    \hypertarget{neural-networks-for-handwritten-digit-recognition-multiclass}{%
\section{Neural Networks for Handwritten Digit Recognition,
Multiclass}\label{neural-networks-for-handwritten-digit-recognition-multiclass}}

In this exercise, you will use a neural network to recognize the
hand-written digits 0-9.

\hypertarget{outline}{%
\section{Outline}\label{outline}}

\begin{itemize}
\tightlist
\item
  Section \ref{1}
\item
  Section \ref{2}
\item
  Section \ref{3}

  \begin{itemize}
  \tightlist
  \item
    Section \ref{ex01}
  \end{itemize}
\item
  Section \ref{4}

  \begin{itemize}
  \tightlist
  \item
    Section \ref{41}
  \item
    Section \ref{42}
  \item
    Section \ref{43}
  \item
    Section \ref{44}
  \item
    Section \ref{45}

    \begin{itemize}
    \tightlist
    \item
      Section \ref{ex02}
    \end{itemize}
  \end{itemize}
\end{itemize}

    \#\# 1 - Packages

First, let's run the cell below to import all the packages that you will
need during this assignment. - \href{https://numpy.org/}{numpy} is the
fundamental package for scientific computing with Python. -
\href{http://matplotlib.org}{matplotlib} is a popular library to plot
graphs in Python. - \href{https://www.tensorflow.org/}{tensorflow} a
popular platform for machine learning.

    \begin{tcolorbox}[breakable, size=fbox, boxrule=1pt, pad at break*=1mm,colback=cellbackground, colframe=cellborder]
\prompt{In}{incolor}{1}{\boxspacing}
\begin{Verbatim}[commandchars=\\\{\}]
\PY{k+kn}{import} \PY{n+nn}{numpy} \PY{k}{as} \PY{n+nn}{np}
\PY{k+kn}{import} \PY{n+nn}{tensorflow} \PY{k}{as} \PY{n+nn}{tf}
\PY{k+kn}{from} \PY{n+nn}{tensorflow}\PY{n+nn}{.}\PY{n+nn}{keras}\PY{n+nn}{.}\PY{n+nn}{models} \PY{k+kn}{import} \PY{n}{Sequential}
\PY{k+kn}{from} \PY{n+nn}{tensorflow}\PY{n+nn}{.}\PY{n+nn}{keras}\PY{n+nn}{.}\PY{n+nn}{layers} \PY{k+kn}{import} \PY{n}{Dense}
\PY{k+kn}{from} \PY{n+nn}{tensorflow}\PY{n+nn}{.}\PY{n+nn}{keras}\PY{n+nn}{.}\PY{n+nn}{activations} \PY{k+kn}{import} \PY{n}{linear}\PY{p}{,} \PY{n}{relu}\PY{p}{,} \PY{n}{sigmoid}
\PY{o}{\PYZpc{}}\PY{k}{matplotlib} widget
\PY{k+kn}{import} \PY{n+nn}{matplotlib}\PY{n+nn}{.}\PY{n+nn}{pyplot} \PY{k}{as} \PY{n+nn}{plt}
\PY{n}{plt}\PY{o}{.}\PY{n}{style}\PY{o}{.}\PY{n}{use}\PY{p}{(}\PY{l+s+s1}{\PYZsq{}}\PY{l+s+s1}{./deeplearning.mplstyle}\PY{l+s+s1}{\PYZsq{}}\PY{p}{)}

\PY{k+kn}{import} \PY{n+nn}{logging}
\PY{n}{logging}\PY{o}{.}\PY{n}{getLogger}\PY{p}{(}\PY{l+s+s2}{\PYZdq{}}\PY{l+s+s2}{tensorflow}\PY{l+s+s2}{\PYZdq{}}\PY{p}{)}\PY{o}{.}\PY{n}{setLevel}\PY{p}{(}\PY{n}{logging}\PY{o}{.}\PY{n}{ERROR}\PY{p}{)}
\PY{n}{tf}\PY{o}{.}\PY{n}{autograph}\PY{o}{.}\PY{n}{set\PYZus{}verbosity}\PY{p}{(}\PY{l+m+mi}{0}\PY{p}{)}

\PY{k+kn}{from} \PY{n+nn}{public\PYZus{}tests} \PY{k+kn}{import} \PY{o}{*} 

\PY{k+kn}{from} \PY{n+nn}{autils} \PY{k+kn}{import} \PY{o}{*}
\PY{k+kn}{from} \PY{n+nn}{lab\PYZus{}utils\PYZus{}softmax} \PY{k+kn}{import} \PY{n}{plt\PYZus{}softmax}
\PY{n}{np}\PY{o}{.}\PY{n}{set\PYZus{}printoptions}\PY{p}{(}\PY{n}{precision}\PY{o}{=}\PY{l+m+mi}{2}\PY{p}{)}
\end{Verbatim}
\end{tcolorbox}

    \#\# 2 - ReLU Activation This week, a new activation was introduced, the
Rectified Linear Unit (ReLU).
\[ a = max(0,z) \quad\quad\text {# ReLU function} \]

    \begin{tcolorbox}[breakable, size=fbox, boxrule=1pt, pad at break*=1mm,colback=cellbackground, colframe=cellborder]
\prompt{In}{incolor}{2}{\boxspacing}
\begin{Verbatim}[commandchars=\\\{\}]
\PY{n}{plt\PYZus{}act\PYZus{}trio}\PY{p}{(}\PY{p}{)}
\end{Verbatim}
\end{tcolorbox}

    
    \begin{verbatim}
Canvas(toolbar=Toolbar(toolitems=[('Home', 'Reset original view', 'home', 'home'), ('Back', 'Back to previous …
    \end{verbatim}

    
    The example from the lecture on the right shows an application of the
ReLU. In this example, the derived ``awareness'' feature is not binary
but has a continuous range of values. The sigmoid is best for on/off or
binary situations. The ReLU provides a continuous linear relationship.
Additionally it has an `off' range where the output is zero.\\
The ``off'' feature makes the ReLU a Non-Linear activation. Why is this
needed? This enables multiple units to contribute to to the resulting
function without interfering. This is examined more in the supporting
optional lab.

    \#\# 3 - Softmax Function A multiclass neural network generates N
outputs. One output is selected as the predicted answer. In the output
layer, a vector \(\mathbf{z}\) is generated by a linear function which
is fed into a softmax function. The softmax function converts
\(\mathbf{z}\) into a probability distribution as described below. After
applying softmax, each output will be between 0 and 1 and the outputs
will sum to 1. They can be interpreted as probabilities. The larger
inputs to the softmax will correspond to larger output probabilities.

    The softmax function can be written:
\[a_j = \frac{e^{z_j}}{ \sum_{k=0}^{N-1}{e^{z_k} }} \tag{1}\]

Where \(z = \mathbf{w} \cdot \mathbf{x} + b\) and N is the number of
feature/categories in the output layer.

    \#\#\# Exercise 1 Let's create a NumPy implementation:

    \begin{tcolorbox}[breakable, size=fbox, boxrule=1pt, pad at break*=1mm,colback=cellbackground, colframe=cellborder]
\prompt{In}{incolor}{3}{\boxspacing}
\begin{Verbatim}[commandchars=\\\{\}]
\PY{c+c1}{\PYZsh{} UNQ\PYZus{}C1}
\PY{c+c1}{\PYZsh{} GRADED CELL: my\PYZus{}softmax}

\PY{k}{def} \PY{n+nf}{my\PYZus{}softmax}\PY{p}{(}\PY{n}{z}\PY{p}{)}\PY{p}{:}  
    \PY{l+s+sd}{\PYZdq{}\PYZdq{}\PYZdq{} Softmax converts a vector of values to a probability distribution.}
\PY{l+s+sd}{    Args:}
\PY{l+s+sd}{      z (ndarray (N,))  : input data, N features}
\PY{l+s+sd}{    Returns:}
\PY{l+s+sd}{      a (ndarray (N,))  : softmax of z}
\PY{l+s+sd}{    \PYZdq{}\PYZdq{}\PYZdq{}}    
    \PY{c+c1}{\PYZsh{}\PYZsh{}\PYZsh{} START CODE HERE \PYZsh{}\PYZsh{}\PYZsh{} }
    \PY{n}{ez} \PY{o}{=} \PY{n}{np}\PY{o}{.}\PY{n}{exp}\PY{p}{(}\PY{n}{z}\PY{p}{)}
    \PY{n}{a} \PY{o}{=} \PY{n}{ez}\PY{o}{/}\PY{n}{np}\PY{o}{.}\PY{n}{sum}\PY{p}{(}\PY{n}{ez}\PY{p}{)}
    \PY{c+c1}{\PYZsh{}\PYZsh{}\PYZsh{} END CODE HERE \PYZsh{}\PYZsh{}\PYZsh{} }
    \PY{k}{return} \PY{n}{a}
\end{Verbatim}
\end{tcolorbox}

    \begin{tcolorbox}[breakable, size=fbox, boxrule=1pt, pad at break*=1mm,colback=cellbackground, colframe=cellborder]
\prompt{In}{incolor}{4}{\boxspacing}
\begin{Verbatim}[commandchars=\\\{\}]
\PY{n}{z} \PY{o}{=} \PY{n}{np}\PY{o}{.}\PY{n}{array}\PY{p}{(}\PY{p}{[}\PY{l+m+mf}{1.}\PY{p}{,} \PY{l+m+mf}{2.}\PY{p}{,} \PY{l+m+mf}{3.}\PY{p}{,} \PY{l+m+mf}{4.}\PY{p}{]}\PY{p}{)}
\PY{n}{a} \PY{o}{=} \PY{n}{my\PYZus{}softmax}\PY{p}{(}\PY{n}{z}\PY{p}{)}
\PY{n}{atf} \PY{o}{=} \PY{n}{tf}\PY{o}{.}\PY{n}{nn}\PY{o}{.}\PY{n}{softmax}\PY{p}{(}\PY{n}{z}\PY{p}{)}
\PY{n+nb}{print}\PY{p}{(}\PY{l+s+sa}{f}\PY{l+s+s2}{\PYZdq{}}\PY{l+s+s2}{my\PYZus{}softmax(z):         }\PY{l+s+si}{\PYZob{}}\PY{n}{a}\PY{l+s+si}{\PYZcb{}}\PY{l+s+s2}{\PYZdq{}}\PY{p}{)}
\PY{n+nb}{print}\PY{p}{(}\PY{l+s+sa}{f}\PY{l+s+s2}{\PYZdq{}}\PY{l+s+s2}{tensorflow softmax(z): }\PY{l+s+si}{\PYZob{}}\PY{n}{atf}\PY{l+s+si}{\PYZcb{}}\PY{l+s+s2}{\PYZdq{}}\PY{p}{)}

\PY{c+c1}{\PYZsh{} BEGIN UNIT TEST  }
\PY{n}{test\PYZus{}my\PYZus{}softmax}\PY{p}{(}\PY{n}{my\PYZus{}softmax}\PY{p}{)}
\PY{c+c1}{\PYZsh{} END UNIT TEST  }
\end{Verbatim}
\end{tcolorbox}

    \begin{Verbatim}[commandchars=\\\{\}]
my\_softmax(z):         [0.03 0.09 0.24 0.64]
tensorflow softmax(z): [0.03 0.09 0.24 0.64]
\textcolor{ansi-green-intense}{ All tests passed.}
    \end{Verbatim}

    Click for hints One implementation uses for loop to first build the
denominator and then a second loop to calculate each output.

\begin{Shaded}
\begin{Highlighting}[]
\KeywordTok{def}\NormalTok{ my\_softmax(z):  }
\NormalTok{    N }\OperatorTok{=} \BuiltInTok{len}\NormalTok{(z)}
\NormalTok{    a }\OperatorTok{=}                     \CommentTok{\# initialize a to zeros }
\NormalTok{    ez\_sum }\OperatorTok{=}                \CommentTok{\# initialize sum to zero}
    \ControlFlowTok{for}\NormalTok{ k }\KeywordTok{in} \BuiltInTok{range}\NormalTok{(N):      }\CommentTok{\# loop over number of outputs             }
\NormalTok{        ez\_sum }\OperatorTok{+=}           \CommentTok{\# sum exp(z[k]) to build the shared denominator      }
    \ControlFlowTok{for}\NormalTok{ j }\KeywordTok{in} \BuiltInTok{range}\NormalTok{(N):      }\CommentTok{\# loop over number of outputs again                }
\NormalTok{        a[j] }\OperatorTok{=}              \CommentTok{\# divide each the exp of each output by the denominator   }
    \ControlFlowTok{return}\NormalTok{(a)}
\end{Highlighting}
\end{Shaded}

Click for code

\begin{Shaded}
\begin{Highlighting}[]
\KeywordTok{def}\NormalTok{ my\_softmax(z):  }
\NormalTok{    N }\OperatorTok{=} \BuiltInTok{len}\NormalTok{(z)}
\NormalTok{    a }\OperatorTok{=}\NormalTok{ np.zeros(N)}
\NormalTok{    ez\_sum }\OperatorTok{=} \DecValTok{0}
    \ControlFlowTok{for}\NormalTok{ k }\KeywordTok{in} \BuiltInTok{range}\NormalTok{(N):                }
\NormalTok{        ez\_sum }\OperatorTok{+=}\NormalTok{ np.exp(z[k])       }
    \ControlFlowTok{for}\NormalTok{ j }\KeywordTok{in} \BuiltInTok{range}\NormalTok{(N):                }
\NormalTok{        a[j] }\OperatorTok{=}\NormalTok{ np.exp(z[j])}\OperatorTok{/}\NormalTok{ez\_sum   }
    \ControlFlowTok{return}\NormalTok{(a)}

\NormalTok{Or, a vector implementation:}

\KeywordTok{def}\NormalTok{ my\_softmax(z):  }
\NormalTok{    ez }\OperatorTok{=}\NormalTok{ np.exp(z)              }
\NormalTok{    a }\OperatorTok{=}\NormalTok{ ez}\OperatorTok{/}\NormalTok{np.}\BuiltInTok{sum}\NormalTok{(ez)           }
    \ControlFlowTok{return}\NormalTok{(a)}
\end{Highlighting}
\end{Shaded}

    Below, vary the values of the \texttt{z} inputs. Note in particular how
the exponential in the numerator magnifies small differences in the
values. Note as well that the output values sum to one.

    \begin{tcolorbox}[breakable, size=fbox, boxrule=1pt, pad at break*=1mm,colback=cellbackground, colframe=cellborder]
\prompt{In}{incolor}{5}{\boxspacing}
\begin{Verbatim}[commandchars=\\\{\}]
\PY{n}{plt}\PY{o}{.}\PY{n}{close}\PY{p}{(}\PY{l+s+s2}{\PYZdq{}}\PY{l+s+s2}{all}\PY{l+s+s2}{\PYZdq{}}\PY{p}{)}
\PY{n}{plt\PYZus{}softmax}\PY{p}{(}\PY{n}{my\PYZus{}softmax}\PY{p}{)}
\end{Verbatim}
\end{tcolorbox}

    
    \begin{verbatim}
Canvas(toolbar=Toolbar(toolitems=[('Home', 'Reset original view', 'home', 'home'), ('Back', 'Back to previous …
    \end{verbatim}

    
    \#\# 4 - Neural Networks

In last weeks assignment, you implemented a neural network to do binary
classification. This week you will extend that to multiclass
classification. This will utilize the softmax activation.

\#\#\# 4.1 Problem Statement

In this exercise, you will use a neural network to recognize ten
handwritten digits, 0-9. This is a multiclass classification task where
one of n choices is selected. Automated handwritten digit recognition is
widely used today - from recognizing zip codes (postal codes) on mail
envelopes to recognizing amounts written on bank checks.

\#\#\# 4.2 Dataset

You will start by loading the dataset for this task. - The
\texttt{load\_data()} function shown below loads the data into variables
\texttt{X} and \texttt{y}

\begin{itemize}
\item
  The data set contains 5000 training examples of handwritten digits
  \(^1\).

  \begin{itemize}
  \tightlist
  \item
    Each training example is a 20-pixel x 20-pixel grayscale image of
    the digit.

    \begin{itemize}
    \tightlist
    \item
      Each pixel is represented by a floating-point number indicating
      the grayscale intensity at that location.
    \item
      The 20 by 20 grid of pixels is ``unrolled'' into a 400-dimensional
      vector.
    \item
      Each training examples becomes a single row in our data matrix
      \texttt{X}.
    \item
      This gives us a 5000 x 400 matrix \texttt{X} where every row is a
      training example of a handwritten digit image.
    \end{itemize}
  \end{itemize}
\end{itemize}

\[X = 
\left(\begin{array}{cc} 
--- (x^{(1)}) --- \\
--- (x^{(2)}) --- \\
\vdots \\ 
--- (x^{(m)}) --- 
\end{array}\right)\]

\begin{itemize}
\tightlist
\item
  The second part of the training set is a 5000 x 1 dimensional vector
  \texttt{y} that contains labels for the training set

  \begin{itemize}
  \tightlist
  \item
    \texttt{y\ =\ 0} if the image is of the digit \texttt{0},
    \texttt{y\ =\ 4} if the image is of the digit \texttt{4} and so on.
  \end{itemize}
\end{itemize}

\(^1\) This is a subset of the MNIST handwritten digit dataset
(http://yann.lecun.com/exdb/mnist/)

    \begin{tcolorbox}[breakable, size=fbox, boxrule=1pt, pad at break*=1mm,colback=cellbackground, colframe=cellborder]
\prompt{In}{incolor}{6}{\boxspacing}
\begin{Verbatim}[commandchars=\\\{\}]
\PY{c+c1}{\PYZsh{} load dataset}
\PY{n}{X}\PY{p}{,} \PY{n}{y} \PY{o}{=} \PY{n}{load\PYZus{}data}\PY{p}{(}\PY{p}{)}
\end{Verbatim}
\end{tcolorbox}

    \hypertarget{view-the-variables}{%
\paragraph{4.2.1 View the variables}\label{view-the-variables}}

Let's get more familiar with your dataset.\\
- A good place to start is to print out each variable and see what it
contains.

The code below prints the first element in the variables \texttt{X} and
\texttt{y}.

    \begin{tcolorbox}[breakable, size=fbox, boxrule=1pt, pad at break*=1mm,colback=cellbackground, colframe=cellborder]
\prompt{In}{incolor}{7}{\boxspacing}
\begin{Verbatim}[commandchars=\\\{\}]
\PY{n+nb}{print} \PY{p}{(}\PY{l+s+s1}{\PYZsq{}}\PY{l+s+s1}{The first element of X is: }\PY{l+s+s1}{\PYZsq{}}\PY{p}{,} \PY{n}{X}\PY{p}{[}\PY{l+m+mi}{0}\PY{p}{]}\PY{p}{)}
\end{Verbatim}
\end{tcolorbox}

    \begin{Verbatim}[commandchars=\\\{\}]
The first element of X is:  [ 0.00e+00  0.00e+00  0.00e+00  0.00e+00  0.00e+00
0.00e+00  0.00e+00
  0.00e+00  0.00e+00  0.00e+00  0.00e+00  0.00e+00  0.00e+00  0.00e+00
  0.00e+00  0.00e+00  0.00e+00  0.00e+00  0.00e+00  0.00e+00  0.00e+00
  0.00e+00  0.00e+00  0.00e+00  0.00e+00  0.00e+00  0.00e+00  0.00e+00
  0.00e+00  0.00e+00  0.00e+00  0.00e+00  0.00e+00  0.00e+00  0.00e+00
  0.00e+00  0.00e+00  0.00e+00  0.00e+00  0.00e+00  0.00e+00  0.00e+00
  0.00e+00  0.00e+00  0.00e+00  0.00e+00  0.00e+00  0.00e+00  0.00e+00
  0.00e+00  0.00e+00  0.00e+00  0.00e+00  0.00e+00  0.00e+00  0.00e+00
  0.00e+00  0.00e+00  0.00e+00  0.00e+00  0.00e+00  0.00e+00  0.00e+00
  0.00e+00  0.00e+00  0.00e+00  0.00e+00  8.56e-06  1.94e-06 -7.37e-04
 -8.13e-03 -1.86e-02 -1.87e-02 -1.88e-02 -1.91e-02 -1.64e-02 -3.78e-03
  3.30e-04  1.28e-05  0.00e+00  0.00e+00  0.00e+00  0.00e+00  0.00e+00
  0.00e+00  0.00e+00  1.16e-04  1.20e-04 -1.40e-02 -2.85e-02  8.04e-02
  2.67e-01  2.74e-01  2.79e-01  2.74e-01  2.25e-01  2.78e-02 -7.06e-03
  2.35e-04  0.00e+00  0.00e+00  0.00e+00  0.00e+00  0.00e+00  0.00e+00
  1.28e-17 -3.26e-04 -1.39e-02  8.16e-02  3.83e-01  8.58e-01  1.00e+00
  9.70e-01  9.31e-01  1.00e+00  9.64e-01  4.49e-01 -5.60e-03 -3.78e-03
  0.00e+00  0.00e+00  0.00e+00  0.00e+00  5.11e-06  4.36e-04 -3.96e-03
 -2.69e-02  1.01e-01  6.42e-01  1.03e+00  8.51e-01  5.43e-01  3.43e-01
  2.69e-01  6.68e-01  1.01e+00  9.04e-01  1.04e-01 -1.66e-02  0.00e+00
  0.00e+00  0.00e+00  0.00e+00  2.60e-05 -3.11e-03  7.52e-03  1.78e-01
  7.93e-01  9.66e-01  4.63e-01  6.92e-02 -3.64e-03 -4.12e-02 -5.02e-02
  1.56e-01  9.02e-01  1.05e+00  1.51e-01 -2.16e-02  0.00e+00  0.00e+00
  0.00e+00  5.87e-05 -6.41e-04 -3.23e-02  2.78e-01  9.37e-01  1.04e+00
  5.98e-01 -3.59e-03 -2.17e-02 -4.81e-03  6.17e-05 -1.24e-02  1.55e-01
  9.15e-01  9.20e-01  1.09e-01 -1.71e-02  0.00e+00  0.00e+00  1.56e-04
 -4.28e-04 -2.51e-02  1.31e-01  7.82e-01  1.03e+00  7.57e-01  2.85e-01
  4.87e-03 -3.19e-03  0.00e+00  8.36e-04 -3.71e-02  4.53e-01  1.03e+00
  5.39e-01 -2.44e-03 -4.80e-03  0.00e+00  0.00e+00 -7.04e-04 -1.27e-02
  1.62e-01  7.80e-01  1.04e+00  8.04e-01  1.61e-01 -1.38e-02  2.15e-03
 -2.13e-04  2.04e-04 -6.86e-03  4.32e-04  7.21e-01  8.48e-01  1.51e-01
 -2.28e-02  1.99e-04  0.00e+00  0.00e+00 -9.40e-03  3.75e-02  6.94e-01
  1.03e+00  1.02e+00  8.80e-01  3.92e-01 -1.74e-02 -1.20e-04  5.55e-05
 -2.24e-03 -2.76e-02  3.69e-01  9.36e-01  4.59e-01 -4.25e-02  1.17e-03
  1.89e-05  0.00e+00  0.00e+00 -1.94e-02  1.30e-01  9.80e-01  9.42e-01
  7.75e-01  8.74e-01  2.13e-01 -1.72e-02  0.00e+00  1.10e-03 -2.62e-02
  1.23e-01  8.31e-01  7.27e-01  5.24e-02 -6.19e-03  0.00e+00  0.00e+00
  0.00e+00  0.00e+00 -9.37e-03  3.68e-02  6.99e-01  1.00e+00  6.06e-01
  3.27e-01 -3.22e-02 -4.83e-02 -4.34e-02 -5.75e-02  9.56e-02  7.27e-01
  6.95e-01  1.47e-01 -1.20e-02 -3.03e-04  0.00e+00  0.00e+00  0.00e+00
  0.00e+00 -6.77e-04 -6.51e-03  1.17e-01  4.22e-01  9.93e-01  8.82e-01
  7.46e-01  7.24e-01  7.23e-01  7.20e-01  8.45e-01  8.32e-01  6.89e-02
 -2.78e-02  3.59e-04  7.15e-05  0.00e+00  0.00e+00  0.00e+00  0.00e+00
  1.53e-04  3.17e-04 -2.29e-02 -4.14e-03  3.87e-01  5.05e-01  7.75e-01
  9.90e-01  1.01e+00  1.01e+00  7.38e-01  2.15e-01 -2.70e-02  1.33e-03
  0.00e+00  0.00e+00  0.00e+00  0.00e+00  0.00e+00  0.00e+00  0.00e+00
  0.00e+00  2.36e-04 -2.26e-03 -2.52e-02 -3.74e-02  6.62e-02  2.91e-01
  3.23e-01  3.06e-01  8.76e-02 -2.51e-02  2.37e-04  0.00e+00  0.00e+00
  0.00e+00  0.00e+00  0.00e+00  0.00e+00  0.00e+00  0.00e+00  0.00e+00
  0.00e+00  0.00e+00  6.21e-18  6.73e-04 -1.13e-02 -3.55e-02 -3.88e-02
 -3.71e-02 -1.34e-02  9.91e-04  4.89e-05  0.00e+00  0.00e+00  0.00e+00
  0.00e+00  0.00e+00  0.00e+00  0.00e+00  0.00e+00  0.00e+00  0.00e+00
  0.00e+00  0.00e+00  0.00e+00  0.00e+00  0.00e+00  0.00e+00  0.00e+00
  0.00e+00  0.00e+00  0.00e+00  0.00e+00  0.00e+00  0.00e+00  0.00e+00
  0.00e+00  0.00e+00  0.00e+00  0.00e+00  0.00e+00  0.00e+00  0.00e+00
  0.00e+00  0.00e+00  0.00e+00  0.00e+00  0.00e+00  0.00e+00  0.00e+00
  0.00e+00  0.00e+00  0.00e+00  0.00e+00  0.00e+00  0.00e+00  0.00e+00
  0.00e+00]
    \end{Verbatim}

    \begin{tcolorbox}[breakable, size=fbox, boxrule=1pt, pad at break*=1mm,colback=cellbackground, colframe=cellborder]
\prompt{In}{incolor}{8}{\boxspacing}
\begin{Verbatim}[commandchars=\\\{\}]
\PY{n+nb}{print} \PY{p}{(}\PY{l+s+s1}{\PYZsq{}}\PY{l+s+s1}{The first element of y is: }\PY{l+s+s1}{\PYZsq{}}\PY{p}{,} \PY{n}{y}\PY{p}{[}\PY{l+m+mi}{0}\PY{p}{,}\PY{l+m+mi}{0}\PY{p}{]}\PY{p}{)}
\PY{n+nb}{print} \PY{p}{(}\PY{l+s+s1}{\PYZsq{}}\PY{l+s+s1}{The last element of y is: }\PY{l+s+s1}{\PYZsq{}}\PY{p}{,} \PY{n}{y}\PY{p}{[}\PY{o}{\PYZhy{}}\PY{l+m+mi}{1}\PY{p}{,}\PY{l+m+mi}{0}\PY{p}{]}\PY{p}{)}
\end{Verbatim}
\end{tcolorbox}

    \begin{Verbatim}[commandchars=\\\{\}]
The first element of y is:  0
The last element of y is:  9
    \end{Verbatim}

    \hypertarget{check-the-dimensions-of-your-variables}{%
\paragraph{4.2.2 Check the dimensions of your
variables}\label{check-the-dimensions-of-your-variables}}

Another way to get familiar with your data is to view its dimensions.
Please print the shape of \texttt{X} and \texttt{y} and see how many
training examples you have in your dataset.

    \begin{tcolorbox}[breakable, size=fbox, boxrule=1pt, pad at break*=1mm,colback=cellbackground, colframe=cellborder]
\prompt{In}{incolor}{9}{\boxspacing}
\begin{Verbatim}[commandchars=\\\{\}]
\PY{n+nb}{print} \PY{p}{(}\PY{l+s+s1}{\PYZsq{}}\PY{l+s+s1}{The shape of X is: }\PY{l+s+s1}{\PYZsq{}} \PY{o}{+} \PY{n+nb}{str}\PY{p}{(}\PY{n}{X}\PY{o}{.}\PY{n}{shape}\PY{p}{)}\PY{p}{)}
\PY{n+nb}{print} \PY{p}{(}\PY{l+s+s1}{\PYZsq{}}\PY{l+s+s1}{The shape of y is: }\PY{l+s+s1}{\PYZsq{}} \PY{o}{+} \PY{n+nb}{str}\PY{p}{(}\PY{n}{y}\PY{o}{.}\PY{n}{shape}\PY{p}{)}\PY{p}{)}
\end{Verbatim}
\end{tcolorbox}

    \begin{Verbatim}[commandchars=\\\{\}]
The shape of X is: (5000, 400)
The shape of y is: (5000, 1)
    \end{Verbatim}

    \hypertarget{visualizing-the-data}{%
\paragraph{4.2.3 Visualizing the Data}\label{visualizing-the-data}}

You will begin by visualizing a subset of the training set. - In the
cell below, the code randomly selects 64 rows from \texttt{X}, maps each
row back to a 20 pixel by 20 pixel grayscale image and displays the
images together. - The label for each image is displayed above the image

    \begin{tcolorbox}[breakable, size=fbox, boxrule=1pt, pad at break*=1mm,colback=cellbackground, colframe=cellborder]
\prompt{In}{incolor}{10}{\boxspacing}
\begin{Verbatim}[commandchars=\\\{\}]
\PY{k+kn}{import} \PY{n+nn}{warnings}
\PY{n}{warnings}\PY{o}{.}\PY{n}{simplefilter}\PY{p}{(}\PY{n}{action}\PY{o}{=}\PY{l+s+s1}{\PYZsq{}}\PY{l+s+s1}{ignore}\PY{l+s+s1}{\PYZsq{}}\PY{p}{,} \PY{n}{category}\PY{o}{=}\PY{n+ne}{FutureWarning}\PY{p}{)}
\PY{c+c1}{\PYZsh{} You do not need to modify anything in this cell}

\PY{n}{m}\PY{p}{,} \PY{n}{n} \PY{o}{=} \PY{n}{X}\PY{o}{.}\PY{n}{shape}

\PY{n}{fig}\PY{p}{,} \PY{n}{axes} \PY{o}{=} \PY{n}{plt}\PY{o}{.}\PY{n}{subplots}\PY{p}{(}\PY{l+m+mi}{8}\PY{p}{,}\PY{l+m+mi}{8}\PY{p}{,} \PY{n}{figsize}\PY{o}{=}\PY{p}{(}\PY{l+m+mi}{5}\PY{p}{,}\PY{l+m+mi}{5}\PY{p}{)}\PY{p}{)}
\PY{n}{fig}\PY{o}{.}\PY{n}{tight\PYZus{}layout}\PY{p}{(}\PY{n}{pad}\PY{o}{=}\PY{l+m+mf}{0.13}\PY{p}{,}\PY{n}{rect}\PY{o}{=}\PY{p}{[}\PY{l+m+mi}{0}\PY{p}{,} \PY{l+m+mf}{0.03}\PY{p}{,} \PY{l+m+mi}{1}\PY{p}{,} \PY{l+m+mf}{0.91}\PY{p}{]}\PY{p}{)} \PY{c+c1}{\PYZsh{}[left, bottom, right, top]}

\PY{c+c1}{\PYZsh{}fig.tight\PYZus{}layout(pad=0.5)}
\PY{n}{widgvis}\PY{p}{(}\PY{n}{fig}\PY{p}{)}
\PY{k}{for} \PY{n}{i}\PY{p}{,}\PY{n}{ax} \PY{o+ow}{in} \PY{n+nb}{enumerate}\PY{p}{(}\PY{n}{axes}\PY{o}{.}\PY{n}{flat}\PY{p}{)}\PY{p}{:}
    \PY{c+c1}{\PYZsh{} Select random indices}
    \PY{n}{random\PYZus{}index} \PY{o}{=} \PY{n}{np}\PY{o}{.}\PY{n}{random}\PY{o}{.}\PY{n}{randint}\PY{p}{(}\PY{n}{m}\PY{p}{)}
    
    \PY{c+c1}{\PYZsh{} Select rows corresponding to the random indices and}
    \PY{c+c1}{\PYZsh{} reshape the image}
    \PY{n}{X\PYZus{}random\PYZus{}reshaped} \PY{o}{=} \PY{n}{X}\PY{p}{[}\PY{n}{random\PYZus{}index}\PY{p}{]}\PY{o}{.}\PY{n}{reshape}\PY{p}{(}\PY{p}{(}\PY{l+m+mi}{20}\PY{p}{,}\PY{l+m+mi}{20}\PY{p}{)}\PY{p}{)}\PY{o}{.}\PY{n}{T}
    
    \PY{c+c1}{\PYZsh{} Display the image}
    \PY{n}{ax}\PY{o}{.}\PY{n}{imshow}\PY{p}{(}\PY{n}{X\PYZus{}random\PYZus{}reshaped}\PY{p}{,} \PY{n}{cmap}\PY{o}{=}\PY{l+s+s1}{\PYZsq{}}\PY{l+s+s1}{gray}\PY{l+s+s1}{\PYZsq{}}\PY{p}{)}
    
    \PY{c+c1}{\PYZsh{} Display the label above the image}
    \PY{n}{ax}\PY{o}{.}\PY{n}{set\PYZus{}title}\PY{p}{(}\PY{n}{y}\PY{p}{[}\PY{n}{random\PYZus{}index}\PY{p}{,}\PY{l+m+mi}{0}\PY{p}{]}\PY{p}{)}
    \PY{n}{ax}\PY{o}{.}\PY{n}{set\PYZus{}axis\PYZus{}off}\PY{p}{(}\PY{p}{)}
    \PY{n}{fig}\PY{o}{.}\PY{n}{suptitle}\PY{p}{(}\PY{l+s+s2}{\PYZdq{}}\PY{l+s+s2}{Label, image}\PY{l+s+s2}{\PYZdq{}}\PY{p}{,} \PY{n}{fontsize}\PY{o}{=}\PY{l+m+mi}{14}\PY{p}{)}
\end{Verbatim}
\end{tcolorbox}

    
    \begin{verbatim}
Canvas(toolbar=Toolbar(toolitems=[('Home', 'Reset original view', 'home', 'home'), ('Back', 'Back to previous …
    \end{verbatim}

    
    \#\#\# 4.3 Model representation

The neural network you will use in this assignment is shown in the
figure below. - This has two dense layers with ReLU activations followed
by an output layer with a linear activation. - Recall that our inputs
are pixel values of digit images. - Since the images are of size
\(20\times20\), this gives us \(400\) inputs

    \begin{itemize}
\item
  The parameters have dimensions that are sized for a neural network
  with \(25\) units in layer 1, \(15\) units in layer 2 and \(10\)
  output units in layer 3, one for each digit.

  \begin{itemize}
  \tightlist
  \item
    Recall that the dimensions of these parameters is determined as
    follows:

    \begin{itemize}
    \tightlist
    \item
      If network has \(s_{in}\) units in a layer and \(s_{out}\) units
      in the next layer, then

      \begin{itemize}
      \tightlist
      \item
        \(W\) will be of dimension \(s_{in} \times s_{out}\).
      \item
        \(b\) will be a vector with \(s_{out}\) elements
      \end{itemize}
    \end{itemize}
  \item
    Therefore, the shapes of \texttt{W}, and \texttt{b}, are

    \begin{itemize}
    \tightlist
    \item
      layer1: The shape of \texttt{W1} is (400, 25) and the shape of
      \texttt{b1} is (25,)
    \item
      layer2: The shape of \texttt{W2} is (25, 15) and the shape of
      \texttt{b2} is: (15,)
    \item
      layer3: The shape of \texttt{W3} is (15, 10) and the shape of
      \texttt{b3} is: (10,) \textgreater{}\textbf{Note:} The bias vector
      \texttt{b} could be represented as a 1-D (n,) or 2-D (n,1) array.
      Tensorflow utilizes a 1-D representation and this lab will
      maintain that convention:
    \end{itemize}
  \end{itemize}
\end{itemize}

    \#\#\# 4.4 Tensorflow Model Implementation

    Tensorflow models are built layer by layer. A layer's input dimensions
(\(s_{in}\) above) are calculated for you. You specify a layer's
\emph{output dimensions} and this determines the next layer's input
dimension. The input dimension of the first layer is derived from the
size of the input data specified in the \texttt{model.fit} statement
below. \textgreater{}\textbf{Note:} It is also possible to add an input
layer that specifies the input dimension of the first layer. For
example:\\
\texttt{tf.keras.Input(shape=(400,)),\ \ \ \ \#specify\ input\ shape}~\\
We will include that here to illuminate some model sizing.

    \#\#\# 4.5 Softmax placement As described in the lecture and the
optional softmax lab, numerical stability is improved if the softmax is
grouped with the loss function rather than the output layer during
training. This has implications when \emph{building} the model and
\emph{using} the model.\\
Building:\\
* The final Dense layer should use a `linear' activation. This is
effectively no activation. * The \texttt{model.compile} statement will
indicate this by including \texttt{from\_logits=True}.
\texttt{loss=tf.keras.losses.SparseCategoricalCrossentropy(from\_logits=True)}\\
* This does not impact the form of the target. In the case of
SparseCategorialCrossentropy, the target is the expected digit, 0-9.

Using the model: * The outputs are not probabilities. If output
probabilities are desired, apply a softmax function.

    \#\#\# Exercise 2

Below, using Keras
\href{https://keras.io/guides/sequential_model/}{Sequential model} and
\href{https://keras.io/api/layers/core_layers/dense/}{Dense Layer} with
a ReLU activation to construct the three layer network described above.

    \begin{tcolorbox}[breakable, size=fbox, boxrule=1pt, pad at break*=1mm,colback=cellbackground, colframe=cellborder]
\prompt{In}{incolor}{16}{\boxspacing}
\begin{Verbatim}[commandchars=\\\{\}]
\PY{c+c1}{\PYZsh{} UNQ\PYZus{}C2}
\PY{c+c1}{\PYZsh{} GRADED CELL: Sequential model}
\PY{n}{tf}\PY{o}{.}\PY{n}{random}\PY{o}{.}\PY{n}{set\PYZus{}seed}\PY{p}{(}\PY{l+m+mi}{1234}\PY{p}{)} \PY{c+c1}{\PYZsh{} for consistent results}
\PY{n}{model} \PY{o}{=} \PY{n}{Sequential}\PY{p}{(}
    \PY{p}{[}
        \PY{c+c1}{\PYZsh{}\PYZsh{}\PYZsh{} START CODE HERE \PYZsh{}\PYZsh{}\PYZsh{} }
        \PY{n}{tf}\PY{o}{.}\PY{n}{keras}\PY{o}{.}\PY{n}{Input}\PY{p}{(}\PY{n}{shape}\PY{o}{=}\PY{p}{(}\PY{l+m+mi}{400}\PY{p}{,}\PY{p}{)}\PY{p}{)}\PY{p}{,}
        \PY{n}{Dense}\PY{p}{(}\PY{l+m+mi}{25}\PY{p}{,} \PY{n}{activation} \PY{o}{=} \PY{l+s+s1}{\PYZsq{}}\PY{l+s+s1}{relu}\PY{l+s+s1}{\PYZsq{}}\PY{p}{)}\PY{p}{,}
        \PY{n}{Dense}\PY{p}{(}\PY{l+m+mi}{15}\PY{p}{,} \PY{n}{activation} \PY{o}{=} \PY{l+s+s1}{\PYZsq{}}\PY{l+s+s1}{relu}\PY{l+s+s1}{\PYZsq{}}\PY{p}{)}\PY{p}{,}
        \PY{n}{Dense}\PY{p}{(}\PY{l+m+mi}{10}\PY{p}{,} \PY{n}{activation} \PY{o}{=} \PY{l+s+s1}{\PYZsq{}}\PY{l+s+s1}{linear}\PY{l+s+s1}{\PYZsq{}}\PY{p}{)}
        \PY{c+c1}{\PYZsh{}\PYZsh{}\PYZsh{} END CODE HERE \PYZsh{}\PYZsh{}\PYZsh{} }
    \PY{p}{]}\PY{p}{,} \PY{n}{name} \PY{o}{=} \PY{l+s+s2}{\PYZdq{}}\PY{l+s+s2}{my\PYZus{}model}\PY{l+s+s2}{\PYZdq{}} 
\PY{p}{)}
\end{Verbatim}
\end{tcolorbox}

    \begin{tcolorbox}[breakable, size=fbox, boxrule=1pt, pad at break*=1mm,colback=cellbackground, colframe=cellborder]
\prompt{In}{incolor}{17}{\boxspacing}
\begin{Verbatim}[commandchars=\\\{\}]
\PY{n}{model}\PY{o}{.}\PY{n}{summary}\PY{p}{(}\PY{p}{)}
\end{Verbatim}
\end{tcolorbox}

    \begin{Verbatim}[commandchars=\\\{\}]
Model: "my\_model"
\_\_\_\_\_\_\_\_\_\_\_\_\_\_\_\_\_\_\_\_\_\_\_\_\_\_\_\_\_\_\_\_\_\_\_\_\_\_\_\_\_\_\_\_\_\_\_\_\_\_\_\_\_\_\_\_\_\_\_\_\_\_\_\_\_
 Layer (type)                Output Shape              Param \#
=================================================================
 dense\_6 (Dense)             (None, 25)                10025

 dense\_7 (Dense)             (None, 15)                390

 dense\_8 (Dense)             (None, 10)                160

=================================================================
Total params: 10,575
Trainable params: 10,575
Non-trainable params: 0
\_\_\_\_\_\_\_\_\_\_\_\_\_\_\_\_\_\_\_\_\_\_\_\_\_\_\_\_\_\_\_\_\_\_\_\_\_\_\_\_\_\_\_\_\_\_\_\_\_\_\_\_\_\_\_\_\_\_\_\_\_\_\_\_\_
    \end{Verbatim}

    Expected Output (Click to expand) The \texttt{model.summary()} function
displays a useful summary of the model. Note, the names of the layers
may vary as they are auto-generated unless the name is specified.

\begin{verbatim}
Model: "my_model"
_________________________________________________________________
Layer (type)                 Output Shape              Param #   
=================================================================
L1 (Dense)                   (None, 25)                10025     
_________________________________________________________________
L2 (Dense)                   (None, 15)                390       
_________________________________________________________________
L3 (Dense)                   (None, 10)                160       
=================================================================
Total params: 10,575
Trainable params: 10,575
Non-trainable params: 0
_________________________________________________________________
\end{verbatim}

    Click for hints

\begin{Shaded}
\begin{Highlighting}[]
\NormalTok{tf.random.set\_seed(}\DecValTok{1234}\NormalTok{)}
\NormalTok{model }\OperatorTok{=}\NormalTok{ Sequential(}
\NormalTok{    [               }
        \CommentTok{\#\#\# START CODE HERE }\AlertTok{\#\#\#}\CommentTok{ }
\NormalTok{        tf.keras.Input(shape}\OperatorTok{=}\NormalTok{(}\DecValTok{400}\NormalTok{,)),     }\CommentTok{\# @REPLACE }
\NormalTok{        Dense(}\DecValTok{25}\NormalTok{, activation}\OperatorTok{=}\StringTok{\textquotesingle{}relu\textquotesingle{}}\NormalTok{, name }\OperatorTok{=} \StringTok{"L1"}\NormalTok{), }\CommentTok{\# @REPLACE }
\NormalTok{        Dense(}\DecValTok{15}\NormalTok{, activation}\OperatorTok{=}\StringTok{\textquotesingle{}relu\textquotesingle{}}\NormalTok{,  name }\OperatorTok{=} \StringTok{"L2"}\NormalTok{), }\CommentTok{\# @REPLACE  }
\NormalTok{        Dense(}\DecValTok{10}\NormalTok{, activation}\OperatorTok{=}\StringTok{\textquotesingle{}linear\textquotesingle{}}\NormalTok{, name }\OperatorTok{=} \StringTok{"L3"}\NormalTok{),  }\CommentTok{\# @REPLACE }
        \CommentTok{\#\#\# }\RegionMarkerTok{END}\CommentTok{ CODE HERE }\AlertTok{\#\#\#}\CommentTok{ }
\NormalTok{    ], name }\OperatorTok{=} \StringTok{"my\_model"} 
\NormalTok{)}
\end{Highlighting}
\end{Shaded}

    \begin{tcolorbox}[breakable, size=fbox, boxrule=1pt, pad at break*=1mm,colback=cellbackground, colframe=cellborder]
\prompt{In}{incolor}{18}{\boxspacing}
\begin{Verbatim}[commandchars=\\\{\}]
\PY{c+c1}{\PYZsh{} BEGIN UNIT TEST     }
\PY{n}{test\PYZus{}model}\PY{p}{(}\PY{n}{model}\PY{p}{,} \PY{l+m+mi}{10}\PY{p}{,} \PY{l+m+mi}{400}\PY{p}{)}
\PY{c+c1}{\PYZsh{} END UNIT TEST     }
\end{Verbatim}
\end{tcolorbox}

    \begin{Verbatim}[commandchars=\\\{\}]
\textcolor{ansi-green-intense}{All tests passed!}
    \end{Verbatim}

    The parameter counts shown in the summary correspond to the number of
elements in the weight and bias arrays as shown below.

    Let's further examine the weights to verify that tensorflow produced the
same dimensions as we calculated above.

    \begin{tcolorbox}[breakable, size=fbox, boxrule=1pt, pad at break*=1mm,colback=cellbackground, colframe=cellborder]
\prompt{In}{incolor}{19}{\boxspacing}
\begin{Verbatim}[commandchars=\\\{\}]
\PY{p}{[}\PY{n}{layer1}\PY{p}{,} \PY{n}{layer2}\PY{p}{,} \PY{n}{layer3}\PY{p}{]} \PY{o}{=} \PY{n}{model}\PY{o}{.}\PY{n}{layers}
\end{Verbatim}
\end{tcolorbox}

    \begin{tcolorbox}[breakable, size=fbox, boxrule=1pt, pad at break*=1mm,colback=cellbackground, colframe=cellborder]
\prompt{In}{incolor}{20}{\boxspacing}
\begin{Verbatim}[commandchars=\\\{\}]
\PY{c+c1}{\PYZsh{}\PYZsh{}\PYZsh{}\PYZsh{} Examine Weights shapes}
\PY{n}{W1}\PY{p}{,}\PY{n}{b1} \PY{o}{=} \PY{n}{layer1}\PY{o}{.}\PY{n}{get\PYZus{}weights}\PY{p}{(}\PY{p}{)}
\PY{n}{W2}\PY{p}{,}\PY{n}{b2} \PY{o}{=} \PY{n}{layer2}\PY{o}{.}\PY{n}{get\PYZus{}weights}\PY{p}{(}\PY{p}{)}
\PY{n}{W3}\PY{p}{,}\PY{n}{b3} \PY{o}{=} \PY{n}{layer3}\PY{o}{.}\PY{n}{get\PYZus{}weights}\PY{p}{(}\PY{p}{)}
\PY{n+nb}{print}\PY{p}{(}\PY{l+s+sa}{f}\PY{l+s+s2}{\PYZdq{}}\PY{l+s+s2}{W1 shape = }\PY{l+s+si}{\PYZob{}}\PY{n}{W1}\PY{o}{.}\PY{n}{shape}\PY{l+s+si}{\PYZcb{}}\PY{l+s+s2}{, b1 shape = }\PY{l+s+si}{\PYZob{}}\PY{n}{b1}\PY{o}{.}\PY{n}{shape}\PY{l+s+si}{\PYZcb{}}\PY{l+s+s2}{\PYZdq{}}\PY{p}{)}
\PY{n+nb}{print}\PY{p}{(}\PY{l+s+sa}{f}\PY{l+s+s2}{\PYZdq{}}\PY{l+s+s2}{W2 shape = }\PY{l+s+si}{\PYZob{}}\PY{n}{W2}\PY{o}{.}\PY{n}{shape}\PY{l+s+si}{\PYZcb{}}\PY{l+s+s2}{, b2 shape = }\PY{l+s+si}{\PYZob{}}\PY{n}{b2}\PY{o}{.}\PY{n}{shape}\PY{l+s+si}{\PYZcb{}}\PY{l+s+s2}{\PYZdq{}}\PY{p}{)}
\PY{n+nb}{print}\PY{p}{(}\PY{l+s+sa}{f}\PY{l+s+s2}{\PYZdq{}}\PY{l+s+s2}{W3 shape = }\PY{l+s+si}{\PYZob{}}\PY{n}{W3}\PY{o}{.}\PY{n}{shape}\PY{l+s+si}{\PYZcb{}}\PY{l+s+s2}{, b3 shape = }\PY{l+s+si}{\PYZob{}}\PY{n}{b3}\PY{o}{.}\PY{n}{shape}\PY{l+s+si}{\PYZcb{}}\PY{l+s+s2}{\PYZdq{}}\PY{p}{)}
\end{Verbatim}
\end{tcolorbox}

    \begin{Verbatim}[commandchars=\\\{\}]
W1 shape = (400, 25), b1 shape = (25,)
W2 shape = (25, 15), b2 shape = (15,)
W3 shape = (15, 10), b3 shape = (10,)
    \end{Verbatim}

    \textbf{Expected Output}

\begin{verbatim}
W1 shape = (400, 25), b1 shape = (25,)  
W2 shape = (25, 15), b2 shape = (15,)  
W3 shape = (15, 1), b3 shape = (10,)
\end{verbatim}

    The following code: * defines a loss function,
\texttt{SparseCategoricalCrossentropy} and indicates the softmax should
be included with the loss calculation by adding
\texttt{from\_logits=True}) * defines an optimizer. A popular choice is
Adaptive Moment (Adam) which was described in lecture.

    \begin{tcolorbox}[breakable, size=fbox, boxrule=1pt, pad at break*=1mm,colback=cellbackground, colframe=cellborder]
\prompt{In}{incolor}{21}{\boxspacing}
\begin{Verbatim}[commandchars=\\\{\}]
\PY{n}{model}\PY{o}{.}\PY{n}{compile}\PY{p}{(}
    \PY{n}{loss}\PY{o}{=}\PY{n}{tf}\PY{o}{.}\PY{n}{keras}\PY{o}{.}\PY{n}{losses}\PY{o}{.}\PY{n}{SparseCategoricalCrossentropy}\PY{p}{(}\PY{n}{from\PYZus{}logits}\PY{o}{=}\PY{k+kc}{True}\PY{p}{)}\PY{p}{,}
    \PY{n}{optimizer}\PY{o}{=}\PY{n}{tf}\PY{o}{.}\PY{n}{keras}\PY{o}{.}\PY{n}{optimizers}\PY{o}{.}\PY{n}{Adam}\PY{p}{(}\PY{n}{learning\PYZus{}rate}\PY{o}{=}\PY{l+m+mf}{0.001}\PY{p}{)}\PY{p}{,}
\PY{p}{)}

\PY{n}{history} \PY{o}{=} \PY{n}{model}\PY{o}{.}\PY{n}{fit}\PY{p}{(}
    \PY{n}{X}\PY{p}{,}\PY{n}{y}\PY{p}{,}
    \PY{n}{epochs}\PY{o}{=}\PY{l+m+mi}{40}
\PY{p}{)}
\end{Verbatim}
\end{tcolorbox}

    \begin{Verbatim}[commandchars=\\\{\}]
Epoch 1/40
157/157 [==============================] - 1s 2ms/step - loss: 1.7094
Epoch 2/40
157/157 [==============================] - 0s 2ms/step - loss: 0.7480
Epoch 3/40
157/157 [==============================] - 0s 2ms/step - loss: 0.4428
Epoch 4/40
157/157 [==============================] - 0s 2ms/step - loss: 0.3463
Epoch 5/40
157/157 [==============================] - 0s 2ms/step - loss: 0.2977
Epoch 6/40
157/157 [==============================] - 0s 2ms/step - loss: 0.2630
Epoch 7/40
157/157 [==============================] - 0s 2ms/step - loss: 0.2361
Epoch 8/40
157/157 [==============================] - 0s 2ms/step - loss: 0.2131
Epoch 9/40
157/157 [==============================] - 0s 2ms/step - loss: 0.2004
Epoch 10/40
157/157 [==============================] - 0s 2ms/step - loss: 0.1805
Epoch 11/40
157/157 [==============================] - 0s 2ms/step - loss: 0.1692
Epoch 12/40
157/157 [==============================] - 0s 2ms/step - loss: 0.1580
Epoch 13/40
157/157 [==============================] - 0s 2ms/step - loss: 0.1507
Epoch 14/40
157/157 [==============================] - 0s 2ms/step - loss: 0.1396
Epoch 15/40
157/157 [==============================] - 0s 2ms/step - loss: 0.1289
Epoch 16/40
157/157 [==============================] - 0s 2ms/step - loss: 0.1255
Epoch 17/40
157/157 [==============================] - 0s 2ms/step - loss: 0.1154
Epoch 18/40
157/157 [==============================] - 0s 2ms/step - loss: 0.1102
Epoch 19/40
157/157 [==============================] - 0s 2ms/step - loss: 0.1016
Epoch 20/40
157/157 [==============================] - 0s 2ms/step - loss: 0.0970
Epoch 21/40
157/157 [==============================] - 0s 2ms/step - loss: 0.0926
Epoch 22/40
157/157 [==============================] - 0s 2ms/step - loss: 0.0891
Epoch 23/40
157/157 [==============================] - 0s 2ms/step - loss: 0.0828
Epoch 24/40
157/157 [==============================] - 0s 2ms/step - loss: 0.0785
Epoch 25/40
157/157 [==============================] - 0s 2ms/step - loss: 0.0755
Epoch 26/40
157/157 [==============================] - 0s 2ms/step - loss: 0.0713
Epoch 27/40
157/157 [==============================] - 0s 2ms/step - loss: 0.0701
Epoch 28/40
157/157 [==============================] - 0s 2ms/step - loss: 0.0617
Epoch 29/40
157/157 [==============================] - 0s 2ms/step - loss: 0.0578
Epoch 30/40
157/157 [==============================] - 0s 2ms/step - loss: 0.0550
Epoch 31/40
157/157 [==============================] - 0s 2ms/step - loss: 0.0511
Epoch 32/40
157/157 [==============================] - 0s 2ms/step - loss: 0.0499
Epoch 33/40
157/157 [==============================] - 0s 2ms/step - loss: 0.0462
Epoch 34/40
157/157 [==============================] - 0s 2ms/step - loss: 0.0437
Epoch 35/40
157/157 [==============================] - 0s 2ms/step - loss: 0.0422
Epoch 36/40
157/157 [==============================] - 0s 2ms/step - loss: 0.0396
Epoch 37/40
157/157 [==============================] - 0s 2ms/step - loss: 0.0366
Epoch 38/40
157/157 [==============================] - 0s 2ms/step - loss: 0.0344
Epoch 39/40
157/157 [==============================] - 0s 2ms/step - loss: 0.0312
Epoch 40/40
157/157 [==============================] - 0s 2ms/step - loss: 0.0294
    \end{Verbatim}

    \hypertarget{epochs-and-batches}{%
\paragraph{Epochs and batches}\label{epochs-and-batches}}

In the \texttt{compile} statement above, the number of \texttt{epochs}
was set to 100. This specifies that the entire data set should be
applied during training 100 times. During training, you see output
describing the progress of training that looks like this:

\begin{verbatim}
Epoch 1/100
157/157 [==============================] - 0s 1ms/step - loss: 2.2770
\end{verbatim}

The first line, \texttt{Epoch\ 1/100}, describes which epoch the model
is currently running. For efficiency, the training data set is broken
into `batches'. The default size of a batch in Tensorflow is 32. There
are 5000 examples in our data set or roughly 157 batches. The notation
on the 2nd line \texttt{157/157\ {[}====} is describing which batch has
been executed.

    \hypertarget{loss-cost}{%
\paragraph{Loss (cost)}\label{loss-cost}}

In course 1, we learned to track the progress of gradient descent by
monitoring the cost. Ideally, the cost will decrease as the number of
iterations of the algorithm increases. Tensorflow refers to the cost as
\texttt{loss}. Above, you saw the loss displayed each epoch as
\texttt{model.fit} was executing. The
\href{https://www.tensorflow.org/api_docs/python/tf/keras/Model}{.fit}
method returns a variety of metrics including the loss. This is captured
in the \texttt{history} variable above. This can be used to examine the
loss in a plot as shown below.

    \begin{tcolorbox}[breakable, size=fbox, boxrule=1pt, pad at break*=1mm,colback=cellbackground, colframe=cellborder]
\prompt{In}{incolor}{22}{\boxspacing}
\begin{Verbatim}[commandchars=\\\{\}]
\PY{n}{plot\PYZus{}loss\PYZus{}tf}\PY{p}{(}\PY{n}{history}\PY{p}{)}
\end{Verbatim}
\end{tcolorbox}

    
    \begin{verbatim}
Canvas(toolbar=Toolbar(toolitems=[('Home', 'Reset original view', 'home', 'home'), ('Back', 'Back to previous …
    \end{verbatim}

    
    \hypertarget{prediction}{%
\paragraph{Prediction}\label{prediction}}

To make a prediction, use Keras \texttt{predict}. Below, X{[}1015{]}
contains an image of a two.

    \begin{tcolorbox}[breakable, size=fbox, boxrule=1pt, pad at break*=1mm,colback=cellbackground, colframe=cellborder]
\prompt{In}{incolor}{23}{\boxspacing}
\begin{Verbatim}[commandchars=\\\{\}]
\PY{n}{image\PYZus{}of\PYZus{}two} \PY{o}{=} \PY{n}{X}\PY{p}{[}\PY{l+m+mi}{1015}\PY{p}{]}
\PY{n}{display\PYZus{}digit}\PY{p}{(}\PY{n}{image\PYZus{}of\PYZus{}two}\PY{p}{)}

\PY{n}{prediction} \PY{o}{=} \PY{n}{model}\PY{o}{.}\PY{n}{predict}\PY{p}{(}\PY{n}{image\PYZus{}of\PYZus{}two}\PY{o}{.}\PY{n}{reshape}\PY{p}{(}\PY{l+m+mi}{1}\PY{p}{,}\PY{l+m+mi}{400}\PY{p}{)}\PY{p}{)}  \PY{c+c1}{\PYZsh{} prediction}

\PY{n+nb}{print}\PY{p}{(}\PY{l+s+sa}{f}\PY{l+s+s2}{\PYZdq{}}\PY{l+s+s2}{ predicting a Two: }\PY{l+s+se}{\PYZbs{}n}\PY{l+s+si}{\PYZob{}}\PY{n}{prediction}\PY{l+s+si}{\PYZcb{}}\PY{l+s+s2}{\PYZdq{}}\PY{p}{)}
\PY{n+nb}{print}\PY{p}{(}\PY{l+s+sa}{f}\PY{l+s+s2}{\PYZdq{}}\PY{l+s+s2}{ Largest Prediction index: }\PY{l+s+si}{\PYZob{}}\PY{n}{np}\PY{o}{.}\PY{n}{argmax}\PY{p}{(}\PY{n}{prediction}\PY{p}{)}\PY{l+s+si}{\PYZcb{}}\PY{l+s+s2}{\PYZdq{}}\PY{p}{)}
\end{Verbatim}
\end{tcolorbox}

    
    \begin{verbatim}
Canvas(toolbar=Toolbar(toolitems=[('Home', 'Reset original view', 'home', 'home'), ('Back', 'Back to previous …
    \end{verbatim}

    
    \begin{Verbatim}[commandchars=\\\{\}]
 predicting a Two:
[[ -7.99  -2.23   0.77  -2.41 -11.66 -11.15  -9.53  -3.36  -4.42  -7.17]]
 Largest Prediction index: 2
    \end{Verbatim}

    The largest output is prediction{[}2{]}, indicating the predicted digit
is a `2'. If the problem only requires a selection, that is sufficient.
Use NumPy
\href{https://numpy.org/doc/stable/reference/generated/numpy.argmax.html}{argmax}
to select it. If the problem requires a probability, a softmax is
required:

    \begin{tcolorbox}[breakable, size=fbox, boxrule=1pt, pad at break*=1mm,colback=cellbackground, colframe=cellborder]
\prompt{In}{incolor}{24}{\boxspacing}
\begin{Verbatim}[commandchars=\\\{\}]
\PY{n}{prediction\PYZus{}p} \PY{o}{=} \PY{n}{tf}\PY{o}{.}\PY{n}{nn}\PY{o}{.}\PY{n}{softmax}\PY{p}{(}\PY{n}{prediction}\PY{p}{)}

\PY{n+nb}{print}\PY{p}{(}\PY{l+s+sa}{f}\PY{l+s+s2}{\PYZdq{}}\PY{l+s+s2}{ predicting a Two. Probability vector: }\PY{l+s+se}{\PYZbs{}n}\PY{l+s+si}{\PYZob{}}\PY{n}{prediction\PYZus{}p}\PY{l+s+si}{\PYZcb{}}\PY{l+s+s2}{\PYZdq{}}\PY{p}{)}
\PY{n+nb}{print}\PY{p}{(}\PY{l+s+sa}{f}\PY{l+s+s2}{\PYZdq{}}\PY{l+s+s2}{Total of predictions: }\PY{l+s+si}{\PYZob{}}\PY{n}{np}\PY{o}{.}\PY{n}{sum}\PY{p}{(}\PY{n}{prediction\PYZus{}p}\PY{p}{)}\PY{l+s+si}{:}\PY{l+s+s2}{0.3f}\PY{l+s+si}{\PYZcb{}}\PY{l+s+s2}{\PYZdq{}}\PY{p}{)}
\end{Verbatim}
\end{tcolorbox}

    \begin{Verbatim}[commandchars=\\\{\}]
 predicting a Two. Probability vector:
[[1.42e-04 4.49e-02 8.98e-01 3.76e-02 3.61e-06 5.97e-06 3.03e-05 1.44e-02
  5.03e-03 3.22e-04]]
Total of predictions: 1.000
    \end{Verbatim}

    To return an integer representing the predicted target, you want the
index of the largest probability. This is accomplished with the Numpy
\href{https://numpy.org/doc/stable/reference/generated/numpy.argmax.html}{argmax}
function.

    \begin{tcolorbox}[breakable, size=fbox, boxrule=1pt, pad at break*=1mm,colback=cellbackground, colframe=cellborder]
\prompt{In}{incolor}{25}{\boxspacing}
\begin{Verbatim}[commandchars=\\\{\}]
\PY{n}{yhat} \PY{o}{=} \PY{n}{np}\PY{o}{.}\PY{n}{argmax}\PY{p}{(}\PY{n}{prediction\PYZus{}p}\PY{p}{)}

\PY{n+nb}{print}\PY{p}{(}\PY{l+s+sa}{f}\PY{l+s+s2}{\PYZdq{}}\PY{l+s+s2}{np.argmax(prediction\PYZus{}p): }\PY{l+s+si}{\PYZob{}}\PY{n}{yhat}\PY{l+s+si}{\PYZcb{}}\PY{l+s+s2}{\PYZdq{}}\PY{p}{)}
\end{Verbatim}
\end{tcolorbox}

    \begin{Verbatim}[commandchars=\\\{\}]
np.argmax(prediction\_p): 2
    \end{Verbatim}

    Let's compare the predictions vs the labels for a random sample of 64
digits. This takes a moment to run.

    \begin{tcolorbox}[breakable, size=fbox, boxrule=1pt, pad at break*=1mm,colback=cellbackground, colframe=cellborder]
\prompt{In}{incolor}{26}{\boxspacing}
\begin{Verbatim}[commandchars=\\\{\}]
\PY{k+kn}{import} \PY{n+nn}{warnings}
\PY{n}{warnings}\PY{o}{.}\PY{n}{simplefilter}\PY{p}{(}\PY{n}{action}\PY{o}{=}\PY{l+s+s1}{\PYZsq{}}\PY{l+s+s1}{ignore}\PY{l+s+s1}{\PYZsq{}}\PY{p}{,} \PY{n}{category}\PY{o}{=}\PY{n+ne}{FutureWarning}\PY{p}{)}
\PY{c+c1}{\PYZsh{} You do not need to modify anything in this cell}

\PY{n}{m}\PY{p}{,} \PY{n}{n} \PY{o}{=} \PY{n}{X}\PY{o}{.}\PY{n}{shape}

\PY{n}{fig}\PY{p}{,} \PY{n}{axes} \PY{o}{=} \PY{n}{plt}\PY{o}{.}\PY{n}{subplots}\PY{p}{(}\PY{l+m+mi}{8}\PY{p}{,}\PY{l+m+mi}{8}\PY{p}{,} \PY{n}{figsize}\PY{o}{=}\PY{p}{(}\PY{l+m+mi}{5}\PY{p}{,}\PY{l+m+mi}{5}\PY{p}{)}\PY{p}{)}
\PY{n}{fig}\PY{o}{.}\PY{n}{tight\PYZus{}layout}\PY{p}{(}\PY{n}{pad}\PY{o}{=}\PY{l+m+mf}{0.13}\PY{p}{,}\PY{n}{rect}\PY{o}{=}\PY{p}{[}\PY{l+m+mi}{0}\PY{p}{,} \PY{l+m+mf}{0.03}\PY{p}{,} \PY{l+m+mi}{1}\PY{p}{,} \PY{l+m+mf}{0.91}\PY{p}{]}\PY{p}{)} \PY{c+c1}{\PYZsh{}[left, bottom, right, top]}
\PY{n}{widgvis}\PY{p}{(}\PY{n}{fig}\PY{p}{)}
\PY{k}{for} \PY{n}{i}\PY{p}{,}\PY{n}{ax} \PY{o+ow}{in} \PY{n+nb}{enumerate}\PY{p}{(}\PY{n}{axes}\PY{o}{.}\PY{n}{flat}\PY{p}{)}\PY{p}{:}
    \PY{c+c1}{\PYZsh{} Select random indices}
    \PY{n}{random\PYZus{}index} \PY{o}{=} \PY{n}{np}\PY{o}{.}\PY{n}{random}\PY{o}{.}\PY{n}{randint}\PY{p}{(}\PY{n}{m}\PY{p}{)}
    
    \PY{c+c1}{\PYZsh{} Select rows corresponding to the random indices and}
    \PY{c+c1}{\PYZsh{} reshape the image}
    \PY{n}{X\PYZus{}random\PYZus{}reshaped} \PY{o}{=} \PY{n}{X}\PY{p}{[}\PY{n}{random\PYZus{}index}\PY{p}{]}\PY{o}{.}\PY{n}{reshape}\PY{p}{(}\PY{p}{(}\PY{l+m+mi}{20}\PY{p}{,}\PY{l+m+mi}{20}\PY{p}{)}\PY{p}{)}\PY{o}{.}\PY{n}{T}
    
    \PY{c+c1}{\PYZsh{} Display the image}
    \PY{n}{ax}\PY{o}{.}\PY{n}{imshow}\PY{p}{(}\PY{n}{X\PYZus{}random\PYZus{}reshaped}\PY{p}{,} \PY{n}{cmap}\PY{o}{=}\PY{l+s+s1}{\PYZsq{}}\PY{l+s+s1}{gray}\PY{l+s+s1}{\PYZsq{}}\PY{p}{)}
    
    \PY{c+c1}{\PYZsh{} Predict using the Neural Network}
    \PY{n}{prediction} \PY{o}{=} \PY{n}{model}\PY{o}{.}\PY{n}{predict}\PY{p}{(}\PY{n}{X}\PY{p}{[}\PY{n}{random\PYZus{}index}\PY{p}{]}\PY{o}{.}\PY{n}{reshape}\PY{p}{(}\PY{l+m+mi}{1}\PY{p}{,}\PY{l+m+mi}{400}\PY{p}{)}\PY{p}{)}
    \PY{n}{prediction\PYZus{}p} \PY{o}{=} \PY{n}{tf}\PY{o}{.}\PY{n}{nn}\PY{o}{.}\PY{n}{softmax}\PY{p}{(}\PY{n}{prediction}\PY{p}{)}
    \PY{n}{yhat} \PY{o}{=} \PY{n}{np}\PY{o}{.}\PY{n}{argmax}\PY{p}{(}\PY{n}{prediction\PYZus{}p}\PY{p}{)}
    
    \PY{c+c1}{\PYZsh{} Display the label above the image}
    \PY{n}{ax}\PY{o}{.}\PY{n}{set\PYZus{}title}\PY{p}{(}\PY{l+s+sa}{f}\PY{l+s+s2}{\PYZdq{}}\PY{l+s+si}{\PYZob{}}\PY{n}{y}\PY{p}{[}\PY{n}{random\PYZus{}index}\PY{p}{,}\PY{l+m+mi}{0}\PY{p}{]}\PY{l+s+si}{\PYZcb{}}\PY{l+s+s2}{,}\PY{l+s+si}{\PYZob{}}\PY{n}{yhat}\PY{l+s+si}{\PYZcb{}}\PY{l+s+s2}{\PYZdq{}}\PY{p}{,}\PY{n}{fontsize}\PY{o}{=}\PY{l+m+mi}{10}\PY{p}{)}
    \PY{n}{ax}\PY{o}{.}\PY{n}{set\PYZus{}axis\PYZus{}off}\PY{p}{(}\PY{p}{)}
\PY{n}{fig}\PY{o}{.}\PY{n}{suptitle}\PY{p}{(}\PY{l+s+s2}{\PYZdq{}}\PY{l+s+s2}{Label, yhat}\PY{l+s+s2}{\PYZdq{}}\PY{p}{,} \PY{n}{fontsize}\PY{o}{=}\PY{l+m+mi}{14}\PY{p}{)}
\PY{n}{plt}\PY{o}{.}\PY{n}{show}\PY{p}{(}\PY{p}{)}
\end{Verbatim}
\end{tcolorbox}

    
    \begin{verbatim}
Canvas(toolbar=Toolbar(toolitems=[('Home', 'Reset original view', 'home', 'home'), ('Back', 'Back to previous …
    \end{verbatim}

    
    Let's look at some of the errors. \textgreater Note: increasing the
number of training epochs can eliminate the errors on this data set.

    \begin{tcolorbox}[breakable, size=fbox, boxrule=1pt, pad at break*=1mm,colback=cellbackground, colframe=cellborder]
\prompt{In}{incolor}{27}{\boxspacing}
\begin{Verbatim}[commandchars=\\\{\}]
\PY{n+nb}{print}\PY{p}{(} \PY{l+s+sa}{f}\PY{l+s+s2}{\PYZdq{}}\PY{l+s+si}{\PYZob{}}\PY{n}{display\PYZus{}errors}\PY{p}{(}\PY{n}{model}\PY{p}{,}\PY{n}{X}\PY{p}{,}\PY{n}{y}\PY{p}{)}\PY{l+s+si}{\PYZcb{}}\PY{l+s+s2}{ errors out of }\PY{l+s+si}{\PYZob{}}\PY{n+nb}{len}\PY{p}{(}\PY{n}{X}\PY{p}{)}\PY{l+s+si}{\PYZcb{}}\PY{l+s+s2}{ images}\PY{l+s+s2}{\PYZdq{}}\PY{p}{)}
\end{Verbatim}
\end{tcolorbox}

    
    \begin{verbatim}
Canvas(toolbar=Toolbar(toolitems=[('Home', 'Reset original view', 'home', 'home'), ('Back', 'Back to previous …
    \end{verbatim}

    
    \begin{Verbatim}[commandchars=\\\{\}]
15 errors out of 5000 images
    \end{Verbatim}

    \hypertarget{congratulations}{%
\subsubsection{Congratulations!}\label{congratulations}}

You have successfully built and utilized a neural network to do
multiclass classification.

    \begin{tcolorbox}[breakable, size=fbox, boxrule=1pt, pad at break*=1mm,colback=cellbackground, colframe=cellborder]
\prompt{In}{incolor}{ }{\boxspacing}
\begin{Verbatim}[commandchars=\\\{\}]

\end{Verbatim}
\end{tcolorbox}


    % Add a bibliography block to the postdoc
    
    
    
\end{document}
